\documentclass[10pt]{exam}

\usepackage{amsmath,amssymb,amsfonts}


\let\oldsqrt\sqrt
% it defines the new \sqrt in terms of the old one
\def\sqrt{\mathpalette\DHLhksqrt}
\def\DHLhksqrt#1#2{\setbox0=\hbox{$#1\oldsqrt{#2\,}$}\dimen0=\ht0
\advance\dimen0-0.2\ht0
\setbox2=\hbox{\vrule height\ht0 depth -\dimen0}%
{\box0\lower0.4pt\box2}}

\oddsidemargin=0in
\evensidemargin=0in
\textwidth=6.3in
\topmargin=-0.5in
\textheight=9in

\parindent=0in
\pagestyle{empty}
\pointsinmargin
\boxedpoints
\begin{document}

%%%(change to appropriate class and semester)
Math 221 Fall 2009 \\
$11/17/2009$

%%%(change to appropriate quiz type and date)
Quiz $\#12$ \hspace{1.9in} {Name:} {\underline {\hspace{2.5in}}}
\vspace{2pc}

\begin{center}
\fbox{\fbox{\parbox{5.5in}{\centering
Show all work clearly and in order. Please box your answers.  10 minutes. }}}
\end{center}
%%%(modify rules, time, points as appropriate)

\vspace{2pc}



%\question
%Indicate the median and the mean on the following graph, using $\tilde{x}$ and $\bar{x}$ respectively.

%figure START---------------------------------------------------------------------------
%\begin{figure}[h]
%\centering
%\includegraphics[scale=0.5]{}
%\end{figure}

%figure END-----------------------------------------------------------------------------

\begin{questions}
(PRACTICE PROBLEM)
Find the most general antiderivative of the function $f(\theta)=5 + \frac{1}{\theta^2} $.
\vfill

\question[3]
Write $\displaystyle \int_{0}^{3} \sin(\sqrt{x}) dx$ as a limit of Riemann sums taking the sample points to be the right endpoints on the subintervals. \underline{\bf DO NOT EVALUATE THE LIMIT}
\vfill

\question[4]
Evaluate $\displaystyle  \int_0^2 3x dx$ as a limit of Riemann sums taking the sample points to be the right endpoints on the subintervals.
\vfill

\end{questions}

\end{document}
