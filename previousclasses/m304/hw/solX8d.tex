\documentclass[10pt]{exam}

\usepackage{amsmath,amssymb,amsfonts}


\oddsidemargin=0in
\evensidemargin=0in
\textwidth=6.3in
\topmargin=-0.5in
\textheight=9in

\parindent=0in
\pagestyle{empty}
\pointsinmargin
\boxedpoints
\begin{document}


{\bf Solution to {\bf X8}(d).}
%%%(modify rules, time, points as appropriate)

The answer is no.  Let's call the solution set from part (c) $W$.  To show $W$ is a subspace of $\mathbb{R}^4$ we would need to show three things (see Theorem 3.3.2).  I will show one of those things fails by coming up with a counter example.

Notice

${\bf a}= \left[ \begin{array}{r} -1/2\\3/4\\0\\0  \end{array} \right]$ is a particular solution to $A{\bf x}={\bf b}$ from part (c).  

So

$2{\bf a}= \left[ \begin{array}{r} -1\\3/2\\0\\0  \end{array} \right]$

But we have $A(2{\bf a})=A \left[ \begin{array}{r} -1\\3/2\\0\\0  \end{array} \right] = \left[ \begin{array}{r} 2\\4\\6\\8  \end{array} \right] \neq \left[ \begin{array}{r} 1\\2\\3\\4  \end{array} \right]$

Hence $W$ is not a subspace of $\mathbb{R}^4$ (since the third property failed which said that for any vector ${\bf x}\in W$ and any $c\in \mathbb{R}$ we need to have $c{\bf x} \in W$ for $W$ to be a subspace).



\end{document}
