
\documentclass[12pt]{article}
%\setlength{\oddsidemargin}{0in}
%\setlength{\evensidemargin}{0in}
%\setlength{\textwidth}{6.5in}
%\setlength{\parindent}{0in}
%\setlength{\parskip}{\baselineskip}
\thispagestyle{empty}
\usepackage{fullpage}
\usepackage{amsmath,amsthm,amsfonts}
\usepackage{graphicx}


\newtheorem*{thm}{Theorem}

\begin{document}

%=======================================================
\begin{center}
{\large \bf Comments for Lecture 5}\\
\bf{2.1.2010}
\end{center}

\begin{center}
{\bf Rank}.  
\end{center}
\noindent Please read p27-28 again.  The rank will be used several times throughout this course.

\begin{center}
{\bf Homogeneous Linear Systems}.  
\end{center}

\noindent Recall that a linear equation is said to be {\it homogeneous} if it is of the the form:

\[ c_1 x_1 + c_2 x_2 + \ldots + c_n x_n = 0\]

\noindent (One could think of this geometrically as a hyperplane through the origin).

\noindent We say that a system of linear equations is {\it homogeneous} if each equation in the system is homogeneous.  So the system would be of the form:

\begin{align*}
c_{11} x_1 + c_{12} x_2 + \ldots + c_{1n} x_n &= 0 \\
c_{21} x_1 + c_{22} x_2 + \ldots + c_{2n} x_n &= 0 \\
\vdots \hspace{3pc} \vdots \hspace{3pc} \vdots \hspace{3pc} &= \vdots \\
c_{m1} x_1 + c_{m2} x_2 + \ldots + c_{mn} x_n &= 0 \\
\end{align*}

\noindent Observe that every homogeneous systems must be consistent, since

\[x_1 =0 , x_2=0, \ldots, x_n=0 \]

\noindent is always a solution to the system (this is called the {\it trivial solution}).

\noindent Also observe that if a homogeneous system of linear equations has a {\it nontrivial solution}

\[x_1 =s_1 , x_2=s_2, \ldots, x_n=s_n \]

\noindent then it must have infinitely many solutions, since

\[x_1 =z s_1 , x_2=z s_2, \ldots, x_n=z s_n \]

\noindent is also a solution for any real number $z$ (Check this yourself!).  From this we get the following theorem:

\begin{thm}  A homogeneous system of linear equations has only the trivial solution or it has infinitely many solutions; there are no other possibilities.
\end{thm}

\noindent Please read 1.5 for more results we had on homogeneous systems of linear equations.
%=======================================================

\end{document}
