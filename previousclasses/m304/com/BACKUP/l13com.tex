
\documentclass[12pt]{article}
%\setlength{\oddsidemargin}{0in}
%\setlength{\evensidemargin}{0in}
%\setlength{\textwidth}{6.5in}
%\setlength{\parindent}{0in}
%\setlength{\parskip}{\baselineskip}
\thispagestyle{empty}
\usepackage{fullpage}
\usepackage{amsmath,amsthm,amsfonts}
\usepackage{graphicx, graphics}
\usepackage[usenames,dvipsnames]{color}

\definecolor{darkyellow}{rgb}{.929412,.8314,0}
\definecolor{brightgreen}{rgb}{.439,.824,.0863}


\newtheorem*{thm}{Theorem}

\begin{document}

%=======================================================
\begin{center}
{\large \bf Comments for Lecture 13}\\
\bf{2.17.2010}
\end{center}

\noindent Here is another example problem using methods from section 2.6.\\

\noindent {\bf Example:}  Consider the linear transformation $T{:}\mathbb{R}^2 \to \mathbb{R}^2$ defined as:

\begin{center}
$T({\bf x})=T\left(  \left[ \begin{array}{c} x_1 \\ x_2  \end{array} \right] \right)=\left[ \begin{array}{c} x_2 \\ x_1  \end{array} \right]$
\end{center}

where ${\bf x}=\left[ \begin{array}{c} x_1 \\ x_2  \end{array} \right]$ is in $\mathbb{R}^2$.

Consider also the linear transformation $S{:}\mathbb{R}^2 \to \mathbb{R}$ defined as:

\begin{center}
$S({\bf y})=S\left(  \left[ \begin{array}{c} y_1 \\ y_2  \end{array} \right] \right)=y_2$
\end{center}

where ${\bf y}=\left[ \begin{array}{c} y_1 \\ y_2  \end{array} \right]$ is in $\mathbb{R}^2$.

{\it Questions:}\\
\noindent(1) Find the associated matrix $A$ such that $T{\bf x} = A{\bf x}$ for any ${\bf x}$ in $\mathbb{R}^2$.\\
\noindent(2) Find the associated matrix $B$ such that $S{\bf y} = B{\bf y}$ for any ${\bf y}$ in $\mathbb{R}^2$.\\
\noindent(3) Does the composition $S(T({\bf x}))$ exist?  If so find the associated matrix $C$ such that $S(T({\bf x})) = C{\bf x}$ for any ${\bf x}$ in $\mathbb{R}^2$.\\
\noindent(4) Does the composition $T(S({\bf y}))$ exist?  If so find the associated matrix $D$ such that $T(S({\bf y})) = D{\bf y}$ for any ${\bf y}$ in $\mathbb{R}^2$.\\


\noindent{\bf Solutions:}\\
\noindent(1) We did this in the comments from lecture 12.

\[ A =\left[ \begin{array}{cc} 0 &1\\1&0  \end{array} \right]  \]


\noindent(2) We did this in the comments from lecture 12.

\[ B =\left[ \begin{array}{cc} 0 &1  \end{array} \right]  \]

\noindent(3) Using {\bf Theorem 2.6.2} the composition $S(T({\bf x}))$ exists and we have $C=BA$.

\[ C=BA = \left[ \begin{array}{cc} 0 &1  \end{array} \right] \left[ \begin{array}{cc} 0 &1\\1&0  \end{array} \right] =  \left[ \begin{array}{cc} 1 &0  \end{array} \right]\] 

(Notice that you could also find this by the method used to solve (1) and (2) but I wanted you to see how we can use theorem 2.6.2 to solve this problem.)\\

\noindent(4) This composition does not exist.  The domain of $T$ is $\mathbb{R}^2$ and the codomain of $S$ is $\mathbb{R}$ (so anything in the image of $S$ will be in $\mathbb{R}$ and cannot be an ``input" for $T$).  To convince even more look at what would happen:

\[ T(S({\bf y})) = T\left(S\left(  \left[ \begin{array}{c} y_1 \\ y_2  \end{array} \right] \right)\right) = T(\text{ }\underbrace{y_2}_{\text{in }\mathbb{R}}\text{ }) = \text{Does not exist}\]

The domain of $T$ is $\mathbb{R}^2$ and $y_2$ is just a real number and not a vector in $\mathbb{R}^2$ so we cannot make sense of $T(y_2)$.



%=======================================================


\end{document}
