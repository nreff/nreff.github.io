
\documentclass[12pt]{article}
%\setlength{\oddsidemargin}{0in}
%\setlength{\evensidemargin}{0in}
%\setlength{\textwidth}{6.5in}
%\setlength{\parindent}{0in}
%\setlength{\parskip}{\baselineskip}
\thispagestyle{empty}
\usepackage{fullpage}
\usepackage{amsmath,amsthm,amsfonts}
\usepackage{graphicx}


\newtheorem*{thm}{Theorem}

\begin{document}

%=======================================================
\begin{center}
{\large \bf Comments for Lecture 6}\\
\bf{2.3.2010}
\end{center}

Start homework 2 as soon as possible!

\begin{center}
{\bf ``The Function"}.  
\end{center}

\noindent Given a system of linear equations with an $m \times n$ coefficient matrix $C$, the system looks like:

\begin{align*}
c_{11} x_1 + c_{12} x_2 + \ldots + c_{1n} x_n &= k_1 \\
c_{21} x_1 + c_{22} x_2 + \ldots + c_{2n} x_n &= k_2 \\
\vdots \hspace{3pc} \vdots \hspace{3pc} \vdots \hspace{3pc} &= \vdots \\
c_{m1} x_1 + c_{m2} x_2 + \ldots + c_{mn} x_n &= k_m \\
\end{align*}

\noindent We talked about how we can actually think of this as a function.  The input here would be the $n$-{\it tuple}: ${\bf x} = (x_1,x_2,\ldots,x_n)$ and the output would be the $m$-{\it tuple}: ${\bf k} = (k_1,k_2,\ldots,k_m)$.  So now we think of $C$ as a function having domain $\mathbb{R}^n$ and codomain $\mathbb{R}^m$ written:

\[C: \mathbb{R}^n \rightarrow \mathbb{R}^m \]

\noindent (NOTE: The dimentions of $C$ as a matrix are $m\times n$)

\noindent and we write

\[C({\bf x})={\bf k} \] 

\noindent As we discussed in class {\bf Theorem 1.6.2} was extremely nice because we were able to determine exactly when $C$ is {\it onto} and when it is {\it one-to-one}.  An amazing thing is that we can get this information for free when we compute {\it rank} of $C$!
%=======================================================

\end{document}
