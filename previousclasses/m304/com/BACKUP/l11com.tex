
\documentclass[12pt]{article}
%\setlength{\oddsidemargin}{0in}
%\setlength{\evensidemargin}{0in}
%\setlength{\textwidth}{6.5in}
%\setlength{\parindent}{0in}
%\setlength{\parskip}{\baselineskip}
\thispagestyle{empty}
\usepackage{fullpage}
\usepackage{amsmath,amsthm,amsfonts}
\usepackage{graphicx, graphics}
\usepackage[usenames,dvipsnames]{color}

\definecolor{darkyellow}{rgb}{.929412,.8314,0}
\definecolor{brightgreen}{rgb}{.439,.824,.0863}


\newtheorem*{thm}{Theorem}

\begin{document}

%=======================================================
\begin{center}
{\large \bf Comments for Lecture 11}\\
\bf{2.11.2010}
\end{center}


\noindent We are continuing our study of the function that an $m \times n$ matrix $C$ determinds from $\mathbb{R}^n$ to $\mathbb{R}^m$.  Again we view this functions as multiplication on the left by $C$:

\begin{center}
\fbox{${\bf x}\mapsto C{\bf x}$}
\end{center}

Where ${\bf x}$ is some arbitary vector in $\mathbb{R}^n$.  Today we showed that this function is indeed a {\it linear transformation}.  This means that for any vectors ${\bf x}$ and ${\bf y}$ in $\mathbb{R}^n$, and for any scalar $k$ in $\mathbb{R}^n$ the following two things are satisfied:

\begin{enumerate}
\item $C({\bf x}+{\bf y})=C({\bf x})+C({\bf y})$
\item $C(k{\bf x})=kC({\bf x})$
\end{enumerate}

\noindent See {\bf Theorem 2.5.1} and {\bf Theorem 2.5.2}.

\noindent So now we can say that if $C$ is an $m \times n$ matrix, then ${\bf x}\mapsto C{\bf x}$ is a linear transformation from $\mathbb{R}^n$ to $\mathbb{R}^m$.\\

\noindent On Friday we will establish quite an amazing converse to this statement:  suppose $T$ is a linear transformation from $\mathbb{R}^n$ to $\mathbb{R}^m$, then there is a unique $m \times n$ matrix $B$ so that $T({\bf x})=B{\bf x}$ for every ${\bf x}$ in $\mathbb{R}^n$.




%=======================================================

\end{document}
