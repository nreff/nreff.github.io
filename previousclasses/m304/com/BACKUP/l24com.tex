
\documentclass[12pt]{article}
%\setlength{\oddsidemargin}{0in}
%\setlength{\evensidemargin}{0in}
%\setlength{\textwidth}{6.5in}
%\setlength{\parindent}{0in}
%\setlength{\parskip}{\baselineskip}
\thispagestyle{empty}
\usepackage{fullpage}
\usepackage{amsmath,amsthm,amsfonts,empheq}
\usepackage{graphicx, graphics}
\usepackage[usenames,dvipsnames]{color}

\definecolor{darkyellow}{rgb}{.929412,.8314,0}
\definecolor{brightgreen}{rgb}{.439,.824,.0863}
\newcommand{\hilight}[1]{\colorbox{yellow}{#1}}

\definecolor{myblue}{rgb}{.6, .93, .6}
\newcommand*\mybluebox[1]{%
\colorbox{myblue}{\hspace{1em}#1\hspace{1em}}}

\newtheorem*{thm}{Theorem}


\begin{document}

%=======================================================
\begin{center}
{\large \bf Comments for Lecture 24}\\
\bf{3.12.2010}
\end{center}


\begin{center}{\LARGE \bf Examples!}.\end{center}


{\bf Example 1:}  Consider the (ordered) set 

\[X = \left( \left[ \begin{array}{r} 1 \\ 1  \\ 0  \end{array} \right], \left[ \begin{array}{r} -1 \\  1 \\ 1  \end{array} \right], \left[ \begin{array}{r}-3 \\ 5  \\ 4  \end{array} \right] \right) \]

\begin{enumerate}
\item Is $X$ linearly indepdent?  If not find a depdence relation on $X$.
\item Does $X$ span $\mathbb{R}^3$ (i.e., is Span($X$)=$\mathbb{R}^3$)?
\item Is $X$ a (ordered) basis of $\mathbb{R}^3$?
\end{enumerate}

{\bf Solution:}

\begin{enumerate}
\item Using the ``test for linear independece" (p129):

Let $A=\left[ \begin{array}{rrr} 1 & -1 & -3  \\ 1 & 1 & 5 \\ 0&1&4  \end{array} \right] $

Putting $A$ into reduced row echelon form we obtain:

\[ A=\left[ \begin{array}{rrr} 1 & -1 & -3  \\ 1 & 1 & 5 \\ 0&1&4  \end{array} \right] \xrightarrow[]{\text{putting } A \text{ in RREF}}  \left[ \begin{array}{rrr} 1 & 0 & 1  \\ 0 & 1 & 4 \\ 0&0&0  \end{array} \right]
\]

So the rank($A$)=2 (number of pivot columns).  So rank($A$)=2$\neq$3 (the number of columns). Hence the answer is \colorbox{myblue}{no}, and $X$ must be linearly depdendent.\\

To find a depdence relation on $X$ we need to find a nontrivial solution to the homogeneous system $A{\bf x} ={\bf 0}$. Well we did all of the work actually, but if you did this from scratch you would set up an augmented matrix for the system and reduce:

\[ \left[ \begin{array}{rrr| r} 1 & -1 & -3 & 0 \\ 1 & 1 & 5 & 0 \\ 0&1&4 &0 \end{array} \right] \xrightarrow[]{\text{putting into RREF}}  \left[ \begin{array}{rrr| r} 1 & 0 & 1 &0 \\ 0 & 1 & 4& 0 \\ 0&0&0& 0  \end{array} \right]
\]

This has a parametric solution of the form:

\begin{align*}
x_1 &= -t\\
x_2 &= -4t\\
x_3 &= t 
\end{align*}

So to get an example of a nontrival solution to the homogenous system  $A{\bf x} ={\bf 0}$ would could set $t=1$ and so $x=\left[ \begin{array}{r} -1 \\ -4  \\ 1  \end{array} \right] $ is a solution.  So we have a depdence relation on $X$:

\begin{empheq}[box=\mybluebox]{align*}
(-1)\left[ \begin{array}{r} 1 \\ 1  \\ 0  \end{array} \right] + (-4)\left[ \begin{array}{r} -1 \\  1 \\ 1  \end{array} \right] + (1)\left[ \begin{array}{r}-3 \\ 5  \\ 4  \end{array} \right] = \left[ \begin{array}{r}0 \\ 0  \\ 0  \end{array} \right] = {\bf 0}
\end{empheq}

\item Using 1. We have rank($A$)$=2\neq3=$ (the number of rows).  Hence the answer is \colorbox{myblue}{no}, and $X$ does not span all of $\mathbb{R}^3$.

\item Using 1. or 2. the answer is \colorbox{myblue}{no}.  Don't forget the definition of a basis! 
\end{enumerate}


{\bf Example 2:}  Let $A=\left[ \begin{array}{rrr} 1 & -1 & -3  \\ 1 & 1 & 5 \\ 0&1&4  \end{array} \right] $

\begin{enumerate}
\item Find a basis $X$ for the column space of $A$.
\item What is the dimension of the column space of $A$?
\item Find a basis $Y$ for the null space of $A$.
\item What is the dimension of the null space of $A$?
\item Find a basis $Z$ for the row space of $A$.
\item What is the dimension of the row space of $A$?
\end{enumerate}

{\bf Solution:} 

\begin{enumerate}
\item We put $A$ into reduced row ecehelon form (RREF):

\[ A=\left[ \begin{array}{rrr} 1 & -1 & -3  \\ 1 & 1 & 5 \\ 0&1&4  \end{array} \right] \xrightarrow[]{\text{putting } A \text{ in RREF}}  \left[ \begin{array}{rrr} 1 & 0 & 1  \\ 0 & 1 & 4 \\ 0&0&0  \end{array} \right]=A'
\]

So this means that columns 1 and 2 are the pivot columns.  The set of pivot columns of $A$ form a basis of the column space of $A$.  Hence,

\begin{empheq}[box=\mybluebox]{align*}
X =  \left( \left[ \begin{array}{r} 1 \\ 1  \\ 0  \end{array} \right], \left[ \begin{array}{r} -1 \\  1 \\ 1  \end{array} \right] \right)
\end{empheq}

\item The dimension of the column space is the number of elements in $X$ hence \colorbox{myblue}{2}.  (Remember we also have that the dimension of the column space = rank($A$) = 2  )
 
\item We find the vector parametric form of the solution set to the homogenous system $A{\bf x}={\bf 0}$:

\[ \left[ \begin{array}{rrr| r} 1 & -1 & -3 & 0 \\ 1 & 1 & 5 & 0 \\ 0&1&4 &0 \end{array} \right] \xrightarrow[]{\text{putting into RREF}}  \left[ \begin{array}{rrr| r} 1 & 0 & 1 &0 \\ 0 & 1 & 4& 0 \\ 0&0&0& 0  \end{array} \right]
\]

This has a parametric solution of the form:

\begin{align*}
x_1 &= -t\\
x_2 &= -4t\\
x_3 &= t 
\end{align*}

or we could write:

\begin{align*}
x_1 &= -x_3\\
x_2 &= -4x_3\\
x_3 &= x_3 
\end{align*}

So the vector parametric form would be:

\[ {\bf x} = \left[ \begin{array}{r} x_1 \\ x_2  \\ x_3  \end{array} \right] =  \left[ \begin{array}{r} -x_3 \\ -4x_3  \\ x_3  \end{array} \right] = x_3 \left[ \begin{array}{r} -1 \\ -4  \\ 1  \end{array} \right] \]

The set of vectors in the vector parametric form of the solution set of $A{\bf x}={\bf 0}$ form a basis for the null space of $A$.  Hence,

\begin{empheq}[box=\mybluebox]{align*}
Y =  \left( \left[ \begin{array}{r} -1 \\ -4  \\ 1  \end{array} \right] \right)
\end{empheq}

\item The dimension of the null space is the number of elements in $Y$ hence \colorbox{myblue}{1}.  (remember we also have that the dimension of the null space = number of non pivot columns = 3 - rank($A$) = 3 - 2 = 1)

\item Using the work from 1. we have that a basis for the row space can be made from the nonzero rows of $A'$ (the reduced row echelon form of $A$).  Hence,

\begin{empheq}[box=\mybluebox]{align*}
Z =  \left( \left[ \begin{array}{rrr} 1&0&1  \end{array} \right] ,  \left[ \begin{array}{rrr} 0&1&4  \end{array} \right]  \right)
\end{empheq}

\item The dimension of the row space is the number of elements in $Z$ hence \colorbox{myblue}{2}.  (Remember we also have that the dimension of the row space = rank($A$) = 2  )
 

\end{enumerate}




%=======================================================


\end{document}
