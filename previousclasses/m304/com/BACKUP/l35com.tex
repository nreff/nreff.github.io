
\documentclass[12pt]{article}
%\setlength{\oddsidemargin}{0in}
%\setlength{\evensidemargin}{0in}
%\setlength{\textwidth}{6.5in}
%\setlength{\parindent}{0in}
%\setlength{\parskip}{\baselineskip}
\thispagestyle{empty}
\usepackage{fullpage}
\usepackage{amsmath,amsthm,amsfonts,empheq}
\usepackage{graphicx, graphics}
\usepackage[usenames,dvipsnames]{color}
\usepackage{booktabs}


\definecolor{darkyellow}{rgb}{.929412,.8314,0}
\definecolor{brightgreen}{rgb}{.439,.824,.0863}
\newcommand{\hilight}[1]{\colorbox{yellow}{#1}}

\definecolor{myblue}{rgb}{.6, .93, .6}
\newcommand*\mybluebox[1]{%
\colorbox{myblue}{\hspace{1em}#1\hspace{1em}}}

\newtheorem*{thm}{Theorem}


\begin{document}

%=======================================================
\begin{center}
{\large \bf Comments for Lecture 35}\\
\bf{4.7.2010}
\end{center}

\begin{center}{\LARGE \bf $\mathbb{R}^n$ vs. $P_n$}.\end{center}


\begin{tabular}{p{4cm}|p{5cm}|p{6cm}|}
\multicolumn{1}{r}{}
 &  \multicolumn{1}{c}{$\mathbb{R}^n$}
 & \multicolumn{1}{c}{$P_n$} \\
\cmidrule{2-3}
Vectors are of the form: & \rule{0cm}{1cm} $\left[ \begin{array}{c} a_1  \\ a_2 \\ \vdots \\ a_n  \end{array} \right] \text{ where each } a_i\in \mathbb{R} $ & $b_0 + b_1 x + b_2 x^2 + \cdots + b_n x^n$ $\text{ where each } b_i\in \mathbb{R} $\\
\cmidrule{2-3}
Example of a vector: &  \rule{0cm}{1cm} $\left[ \begin{array}{c} 5  \\ 0 \\ \vdots \\ 0  \end{array} \right]$ & $1 + 2 x +3x^{n-1}+7x^n$\\
\cmidrule{2-3}
Example of a basis: & $\{ {\bf e}_1 , {\bf e}_2 , \ldots , {\bf e}_n \}$ & $\{ 1 , x , x^2 ,  \ldots , x^n \}$ \\
\cmidrule{2-3}
Example of an ordered basis: & $( {\bf e}_1 , {\bf e}_2 , \ldots , {\bf e}_n )$ & $( 1 , x , x^2 ,  \ldots , x^n )$ \\
\cmidrule{2-3}
Dimension: & $n$ & $n+1$ \\

\cmidrule{2-3}
\end{tabular}





%=======================================================


\end{document}
