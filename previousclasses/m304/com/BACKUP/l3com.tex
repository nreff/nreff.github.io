
\documentclass[12pt]{article}
%\setlength{\oddsidemargin}{0in}
%\setlength{\evensidemargin}{0in}
%\setlength{\textwidth}{6.5in}
%\setlength{\parindent}{0in}
%\setlength{\parskip}{\baselineskip}
\thispagestyle{empty}
\usepackage{fullpage}
\usepackage{amsmath,amsthm,amsfonts}
\usepackage{graphicx}

\begin{document}

%=======================================================
\begin{center}
{\large \bf Comments for Lecture 3}\\
\bf{1.28.2010}
\end{center}

Start working on the homework ASAP.  To get use to some of the definitions that we used today please try to work on (5)1 on page 18.\\

\noindent{\bf Practice problems:}\\
Find the row echelon form of the following augmented matrices:

$$
\left[ \begin {array}{ccc|c} 1&-1&1&2\\ -1&2&-1&1
\\ 2&1&3&0\end {array} \right] 
$$

and

$$
\left[ \begin {array}{ccc|c} 3&1&-1&10\\ 2&1&2&5
\\ -2&2&3&1\end {array} \right]
$$

\noindent{\bf Questions to think about after you try (5)1:}\\

\noindent Is a {\it triangular matrix} always in row echelon form (REF)?\\

\noindent Is a {\it diagonal matrix} always in row echelon form (REF)?\\

\noindent Is a {\it diagonal matrix} always in reduced row echelon form (RREF)?\\

\noindent If the answer was NO for any of the above, then how could we change the question(s) so the answer(s) are yes?\\

\noindent If you are given an arbitrary diagonal matrix, how do you obtain a row equivalent matrix that is in reduced row echelon form (RREF)?\\





%=======================================================

\end{document}
