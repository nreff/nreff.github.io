\documentclass[10pt]{exam}

\usepackage{amsmath,amssymb,amsfonts}


\oddsidemargin=0in
\evensidemargin=0in
\textwidth=6.3in
\topmargin=-0.5in
\textheight=9in

\parindent=0in
\pagestyle{empty}
\pointsinmargin
\boxedpoints
\begin{document}

%%%(change to appropriate class and semester)
Math 304 Spring 2010 \\
$4/9/2010$

%%%(change to appropriate quiz type and date)
Quiz $\#10$ \hspace{1.9in} {Name:} {\underline {\hspace{2.5in}}}
\vspace{.5pc}

\begin{center}
\fbox{\fbox{\parbox{5.5in}{\centering
Due on Monday 4/12/2010 }}}
\end{center}
%%%(modify rules, time, points as appropriate)

\begin{questions}


\question Circle ``True" at each statement that is always true, and circle ``False" at each
statement is not always true.  In the following questions we will always denote $P_n$ as the vector space of polynomials of degree at most $n$.
\begin{choices}
\choice \begin{tabular}{|c|c|}\hline True & False \\ \hline \end{tabular} If $V$ is a finite dimensional vector space then the {\it dimension} of $V$ is the number of vectors in any finite basis of $V$.
\choice \begin{tabular}{|c|c|}\hline True & False \\ \hline \end{tabular} The set $\{{\bf e}_1,{\bf e}_2,{\bf e}_3,{\bf e}_4 \}$ is a basis of $\mathbb{R}^4$.
\choice \begin{tabular}{|c|c|}\hline True & False \\ \hline \end{tabular} The set $\{1,x,x^2,x^3,x^4 \}$ is a basis of $P_4$.
\choice \begin{tabular}{|c|c|}\hline True & False \\ \hline \end{tabular} The dimension of  $\mathbb{R}^4$ is 4.
\choice \begin{tabular}{|c|c|}\hline True & False \\ \hline \end{tabular} The dimension of  $P_4$ is 5.
\choice \begin{tabular}{|c|c|}\hline True & False \\ \hline \end{tabular} There exists an isomorphism from $P_4$ to $\mathbb{R}^4$.
\choice \begin{tabular}{|c|c|}\hline True & False \\ \hline \end{tabular}  $\mathbb{R}^4$ has a basis $X$ such that each vector in  $\mathbb{R}^4$ can be written in more than one way as a linear combination of the elements of $X$.
\choice \begin{tabular}{|c|c|}\hline True & False \\ \hline \end{tabular} $P_4$ has a basis $X$ such that each polynomial (vector) in $P_4$ can be written in more than one way as a linear combination of the elements of $X$.
\choice \begin{tabular}{|c|c|}\hline True & False \\ \hline \end{tabular} The set of functions $\{c_2 x^2 + c_3 x^3 + c_4 x^4 \mid c_2,c_3,c_4 \in \mathbb{R} \}$ is a subspace of $P_4$.
\choice \begin{tabular}{|c|c|}\hline True & False \\ \hline \end{tabular} The set $\{1,1-x,1+x^2\}$ is a basis of $P_2$.
\choice \begin{tabular}{|c|c|}\hline True & False \\ \hline \end{tabular} If $X$ is a collection of vectors in a vector space $W$, then Span($X$) is a subspace of $W$.
\choice \begin{tabular}{|c|c|}\hline True & False \\ \hline \end{tabular} Span$\left(\left \{ \left[ \begin{array}{r} 1 \\ -1 \\ 5\\ \end{array}\right] \right \} \right)$ is a subspace of $\mathbb{R}^3$.
\choice \begin{tabular}{|c|c|}\hline True & False \\ \hline \end{tabular} Span($\{1,1-x\}$) is a subspace of $P_2$.
\choice \begin{tabular}{|c|c|}\hline True & False \\ \hline \end{tabular} Span($\{5\}$) is a subspace of $P_2$.
\choice \begin{tabular}{|c|c|}\hline True & False \\ \hline \end{tabular} If the set $S$ is linearly independent in $P_4$ then $S \cup \{x\}$ is always linearly independent.
\choice \begin{tabular}{|c|c|}\hline True & False \\ \hline \end{tabular} If $S$ is a spanning set of $P_4$ then $S$ always contains the vector (polynomial) 1.  
\choice \begin{tabular}{|c|c|}\hline True & False \\ \hline \end{tabular} If $S$ is a spanning set of $P_4$ then Span($S$) always contains the vector (polynomial) 1.
\choice \begin{tabular}{|c|c|}\hline True & False \\ \hline \end{tabular} A linear transformation is an isomorphism if and only if it is a one-to-one correspondence.
\choice \begin{tabular}{|c|c|}\hline True & False \\ \hline \end{tabular} The linear transformation $T{:}\mathbb{R}^2 \rightarrow \mathbb{R}^2$ with associated matrix $A=\left[\begin{array}{rr} 1 & -2 \\ 2 & 1 \end{array} \right]$ is an isomorphism. 
\choice \begin{tabular}{|c|c|}\hline True & False \\ \hline \end{tabular} Any isomorphism from $\mathbb{R}^4$ to $\mathbb{R}^4$ takes the standard basis $\{{\bf e}_1,{\bf e}_2,{\bf e}_3,{\bf e}_4 \}$ to itself.
\end{choices}
\end{questions}

\end{document}
