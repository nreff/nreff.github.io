\documentclass[12pt]{article}
%\setlength{\oddsidemargin}{0in}
%\setlength{\evensidemargin}{0in}
%\setlength{\textwidth}{6.5in}
%\setlength{\parindent}{0in}
%\setlength{\parskip}{\baselineskip}
\thispagestyle{empty}
%\usepackage{fullpage}

\oddsidemargin=0in

\evensidemargin=0in

\textwidth=6.3in

\topmargin=-0.5in

\textheight=9in

%\parindent=0in

\setlength{\voffset}{0.5 in}

\begin{document}
\begin{center}
{\Large \bf Math 152 - Calculus II}\\
{\bf Fall 2012}
\end{center}

%{\bf Course:}
%\begin{quote}
%Lecture:  MWF 2:00--2:50, CB 204\\
%\end{quote}

{\bf Basic Info:}
\begin{quote}
\begin{tabular}{ll}
Instructor:& Dr.\ Nathan Reff\\
Course Meetings:& MTWF 8:20AM-9:10AM in Myers Hall 228\\
Email:&{\tt reff@alfred.edu}\\
Web Page:&{\tt ?}\\
Office Hours: & ?\\
Text:&{\it Calculus Early Transcendentals, $9^{th}$ Ed.}\\
 & by Anton, Bivens and Davis \texttt{\scriptsize (ISBN-10:\ 9780470183458)} 
\end{tabular}
\end{quote}


{\bf Prerequisite:} You \underline{absolutely} must have a passing grade in Math 151 (Calculus I) or equivalent to be in this course.

{\bf Grading:}  Your final grade will be determined as follows:
\begin{quote}
\begin{center}
\begin{tabular}{lr}
Quizzes and Homework & 30\% \\
Midterm & 30\% \\
Comprehensive Final Examination & 40\% \\
\end{tabular}
\end{center}
\end{quote}

You also have the opportunity to earn participation points which will be added to your course score.  You can earn these points by answering questions, asking relevant questions, etc.  Coming to class is expected and will not get you these participation points alone.  I would like everyone to be a part of the classroom discussions.  

Borderline cases of grades can be adjusted up or down based on your attendance, class participation, homework, and trend (patterns in the grades as the semester progresses, for example, steady improvement is good, but a weak final exam is bad). 


{\bf Seat Number:}  Since this is a large class we will have seating assignments.  Starting on the second day of class where you sit will remain the same throughout the semester.  Anything turned in with your name on it will be accompanied by your seat number.

{\bf Homework:} Homework problems will be assigned daily/weekly and will be collected on exam days.  You must bring all of your homework to every lecture (see quizzes below).  Please make sure your homework is {\it neat} (legible, not torn out of a spiral bound notebook, etc.) and {\it stapled} when you turn it in.  It is {\it very} important that you keep working on problems throughout the course.  There is an old saying that ``math is not a spectator sport" and there is definitely a lot of truth to this.  I recommend working on your own and also with other classmates (but make sure you are turning in your own work).  If you are working on a problem and get stuck, make a note of it and remember to {\it ask questions}.  I encourage {\it everyone} to come to office hours and visit the help room frequently.

Other than assigned problems you should be reading the text every day and keeping up with the pace of the course.  Keep in mind that it your responsibility to read each chapter before an exam.

{\bf Quizzes:} There will at least one quiz each week.  Quizzes may be announced or unannounced.  Quizzes will usually cover lecture material and homework problems.  The questions may even be taken directly from the homework set, or minor perturbations of the homework problems.  Also, there will be homework quizzes where you will just copy exactly what your have written as a homework solution.  These will be 5 minute quizzes of just copying.  You may not look at the problem in the text or have a sheet with the problems written on them.  This will hopefully give you even more encouragement to do the homework.   There will be no make up quizzes.  %The lowest two quizzes will be dropped if you attempt every quiz.


{\bf Exams:} All exams for all sections will be administered at a common class time.  The major exam schedule is as follows:
\begin{quote}
\begin{center}
\begin{tabular}{lr}
Midterm: Planned for Oct 20 in class.\\
Final Exam: Dec 13, 8:30 - 10:30 AM in UU MAN A.
\end{tabular}
\end{center}
\end{quote}
Please see the course website for more details.  Exams will be more challenging than the quizzes so you need to study accordingly.  However doing the homework and reviewing the quizzes is the best way to prepare yourself.  %Practice exams will also be made available on the course website.  

{\bf Quiz/Exam/Final Policy:}
No calculators, cellphones, mp3 players, slide rules, abaci, Addiators, Napier's bones, Difference/Analytical Engines, Pascalinas, Antikythera mechanisms, etc. may be used.  In other words I want you to only use your brain and the hard work you put into this course to earn your grade.  You may not talk to each other in the classroom while other students are working, even if you are done.  Please keep your eyes on your own paper.  Do not look at notes, books, etc. while working. Work through the problem on your own and you will do fine (and save us both a lot of trouble).  

{\bf Cheating and Academic Honesty:}
Cheating of any kind will not be tolerated. It is disrespectful to the University, your classmates and to me.  If you are caught cheating you will receive a 0 on the assignment. Additionally, depending on the severity of the cheating, you may receive an F in the course. Furthermore you will be subject to the University's disciplinary action.

{\bf Attendance Policy:}
Binghamton University has a $75\%$ attendance policy (details of this are located in the bulletin, available at {\tt http://bulletin.binghamton.edu/}).  I will not enforce any additional policy other than this. This course will move rather quickly so I suggest you only miss class for a good reason (meaning an excused absence).  If you must miss a class it is your responsibility to learn the missed material quickly to keep up with the course. 

{\bf Excused Absences:}
If you cannot attend one of the exams you should submit a written reason for your absence {\bf in advance} of the exam date.   I would appreciate letting me know at least 3 days in advance if you are going to miss a class.  In emergency situations you can leave a message for me with the Math Department Secretaries at (607) 777-2147.   The decision to allow make-up exams will be made on a case by case basis, but proper documentation is always necessary. No make-up exams will be given without advance notice.  If you miss a quiz, exam or final with an unexcused absence, you will receive a 0 for that particular assignment. 


{\bf Extra Credit \& ``Curving":}
	I will not curve any quiz, exam or course grade.  I will not be giving anyone extra credit.  This way everyone has the same advantage in the course.  In the rare event of something like an unfair question, I will make appropriate steps to give everyone the same chance to get  credit where it is due.   \\ 

\end{document}

----------------------------------------------------------------------
