\documentclass{article}
\usepackage[utf8]{inputenc}
\usepackage{amsmath,amssymb,amsfonts,amsthm}

\newtheorem{theorem}{Theorem}

\title{MTH 629}
\author{nathan.reff }
\date{March 2018}

\begin{document}

\maketitle

\begin{abstract}
Here we explore some \LaTeX examples.
\end{abstract}

\section*{Introduction}

blah blah blah \\
blah blah blah blah blah blah


% This is how to create a comment in LaTeX
% The $ here puts us into "math mode" and LaTeX will convert this properly.
$\alpha$


a=b=c


% Here \[ will also put is into "display style math mode" where the math is centered and more pronounced.
\[ a=b=c=\sum_{i=1}^n \alpha_i \beta^i \]


% Matrix with square brackets
\[
\begin{bmatrix}
1 & 2 \\
3 & 4
\end{bmatrix}
\]


%Display math mode equation
\[ 
(1-x)^3 = (1-x)(1-2x+x^2) = 1-2x+x^2 -x +2x^2 -x^3
\]


%desplay math mode align.  In the align environment the & symbols will line up, so here the = all line up.  The \\ is a line break.  The * here is just to remove any number labels to the equations.
\begin{align*}
(1-x)^3 &= (1-x)(1-2x+x^2)\\
        &= 1-2x+x^2 -x +2x^2 -x^3    
\end{align*}

$a=b and b=c$ \\

$a=b\text{ and }b=c$\\


$a=b$ and $b=c$


%here is the theorem environment
\begin{theorem}
Everything above QED is true
\end{theorem}
%here is the proof environment
\begin{proof}
QED
\end{proof}


%sections and subsections can be created like this.  The * here will remove a number associated to the section.
\section*{Background}
\subsection*{Graphs}
\subsection*{Matrix Analysis}

\section*{Adjacency Matrices}


\end{document}
