\documentclass[reqno]{amsart}
\usepackage{amsmath,amssymb,amsfonts}
\usepackage{eucal}
\usepackage{graphicx,psfrag}
\usepackage{hyperref}
\usepackage{multirow}
\usepackage{verbatim} 
\usepackage{empheq}
\usepackage{color}

\newtheorem{lem}{Lemma}[section]
\newtheorem{cor}[lem]{Corollary}
\newtheorem{prop}[lem]{Proposition}
\newtheorem{thm}[lem]{Theorem}
\newtheorem{eg}[lem]{Example}

\definecolor{myblue}{rgb}{.8, .8, 1}
\newcommand*\mybluebox[1]{\colorbox{myblue}{\hspace{1em}#1\hspace{1em}}}
\newcommand*\widefbox[1]{\fbox{\hspace{1em}#1\hspace{1em}}}

\numberwithin{equation}{section}

\newcommand\tr{\mathrm{tr}}
\newcommand\bip{\mathrm{bip}}
\newcommand\rank{\operatorname{rank}}
\begin{document}

\title{8.1 Integration by Parts}

\author{Nathan Reff}
\address{Department of Mathematics\\Alfred University\\ Alfred, NY 14802, U.S.A.}
\email{reff@alfred.edu}

\date{\today}
\maketitle

%%%%%%%%%%%%%%%%%%%%%%%%%%%%%%%%%%%%
\section{Integration by Parts}

Recall:
\[ \frac{d}{dx} [f(x)g(x)] = f'(x)g(x)+f(x)g'(x). \]
Integrate both sides
\begin{align*}
f(x)g(x)+C &= \int [f'(x)g(x)+f(x)g'(x)] dx \\
f(x)g(x)+C &= \int f'(x)g(x) dx + \int f(x)g'(x) dx \\
\intertext{so}
\int f(x)g'(x) &= f(x)g(x) - \int f'(x)g(x) dx +C.
\end{align*}
Let's ignore the arbitrary constant since the remaining integral will introduce an abritrary constant anyways.

If we let $u=f(x)$ and $v=g(x)$ then:

\begin{empheq}[box=\widefbox]{align}
 \int u dv = uv - \int v du. 
\end{empheq}

For definite integrals:

\begin{empheq}[box=\widefbox]{align}
 \int_a^b u dv = \left[uv\right]_a^b - \int_a^b v du.
\end{empheq}

\begin{eg} $\int x\cos(x) dx.$
$u=x$, $dv=\cos(x) dx$\\
$du=dx$, $v=\sin(x)$\\
and so 
\begin{align*} \int x \cos(x) dx &= x \sin(x) - \int \sin(x) dx \\
&= x \sin(x) - (-\cos(x)) +C\\
&= x \sin(x) + \cos(x) +C.
\end{align*}
\end{eg}

Sometimes we must do the method twice in the integration.

\begin{eg} $\int x^2 e^x dx$

\begin{center}
  \begin{tabular}{ r||l}
    $u=x^2$ & $dv=e^x dx$ \\ 
    $du=2xdx$ & $v=e^x$\\ 
  \end{tabular}
\end{center}


and so 
\begin{align*} \int  x^2 e^x dx &= x^2 e^x - \int e^x 2x dx \\
&= x^2 e^x - 2\int xe^x  dx. 
\end{align*}
We must use integratin by parts again.

here\\
\begin{center}
  \begin{tabular}{ r||l}
    $u=x$ & $dv=e^x dx$ \\ 
    $du=dx$ & $v=e^x$\\ 
  \end{tabular}
\end{center}

so 
\begin{align*} \int  x^2 e^x dx &= x^2 e^x - 2\int xe^x  dx \\
&=  x^2 e^x - 2 \left[xe^x - \int e^x dx \right] \\
&=  x^2 e^x - 2 \left[xe^x - e^x +C \right] \\
&= x^2 e^x - 2x e^x +2e^x +D.
\end{align*}
\end{eg}

Sometimes we use integration by parts and $dv=du$ and $u=$the rest...

\begin{eg} Consider $\int \ln(x) dx$ or $\int \tan^{-1}(x)dx$.
\end{eg}

Sometimes we use integration by parts and the integral we are solving for appears...then we use algebra!

\begin{eg} $\int e^x \sin(x)dx$.
Let
\begin{center}
  \begin{tabular}{ r||l}
    $u=e^x$ & $dv=\sin(x) dx$ \\ 
    $du=e^xdx$ & $v=-\cos(x)$\\ 
  \end{tabular}
\end{center}

\begin{align*} \int e^x \sin(x)dx &= e^x(-\cos(x)) - \int (-\cos(x))e^x dx \\
&= -e^x\cos(x) + \int e^x\cos(x) dx
\end{align*}
Use integration by parts again:

\begin{center}
  \begin{tabular}{ r||l}
    $u=e^x$ & $dv=\cos(x) dx$ \\ 
    $du=e^xdx$ & $v=\sin(x)$\\ 
  \end{tabular}
\end{center}

\begin{align*} \int e^x \sin(x)dx &= -e^x\cos(x) + \int e^x\cos(x) dx \\
&= -e^x\cos(x) + \left[ e^x \sin(x) - \int \sin(x) e^x dx \right]\\
&= -e^x\cos(x) + e^x \sin(x) - \int e^x \sin(x)  dx.
\end{align*}
Notice we have the integral we are trying to solve on the right.   Bring it over to the left, join the party via algebra.  So we have:
\[2 \int e^x \sin(x)dx =-e^x\cos(x) + e^x \sin(x). \]
Therefore,
 \[\int e^x \sin(x)dx =\frac{-e^x\cos(x) + e^x \sin(x)}{2} +C. \]
\end{eg}

\begin{eg} $\int_1^e \ln(x) dx$.

Let
\begin{center}
  \begin{tabular}{ r||l}
    $u=\ln(x)$ & $dv=dx$ \\ 
    $du=\frac{1}{x}dx$ & $v=x$\\ 
  \end{tabular}
\end{center}

\begin{align*} \int_1^e \ln(x)dx &= \left[x\ln(x)\right]_1^e - \int_1^e x \cdot \frac{1}{x}dx \\
&= [x\ln(x)]_1^e - \int_1^e 1 dx \\
&= [x\ln(x)-x]_1^e \\
&= (e \ln(e) - e) - (1 \ln(1) -1) \\
&= (e-e)-(0-1) = 1. 
\end{align*}
\end{eg}



\end{document}
