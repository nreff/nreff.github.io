\documentclass{article}
\usepackage[utf8]{inputenc}
\usepackage{amsmath,amssymb,amsfonts,amsthm}

\newtheorem{theorem}{Theorem}

\title{MTH 629}
\author{Nate Reff }
\date{\today}

\begin{document}

\maketitle

\begin{abstract}
Here we explore some \LaTeX examples.
\end{abstract}

\section*{Introduction}

$1-x^2$

\[1-x^2\]

% this is a comment!
%%%%%%%%%%%%%%%%%%%%%%%%%%%%%%%%%%%%%%
%%%%%%%%%%%%%%%%%%%%%%%%%%%%%%%%%%%%%%
\[(1-x)^3 = (1-x)(1-2x+x^2)\]
---------
\begin{align}
(1-x)^3 = (1-x)(1-2x+x^2) \label{text}
\end{align}
%
blah blah blah \\
By Equation (\ref{text})
blah blah blah blah blah blah


\begin{itemize} \setlength\itemsep{0em}
    \item Property 1.
    \[ \alpha^2 = \sum_{i=1}^{\infty} a_i \]
    \item Propert 2.
    \item Final thing.
\end{itemize}

\vspace{20pt}
\vfill
TEST \hspace{30pt} SPACING\\
TEST \hfill SPACING


\begin{itemize}
    \item[(I)] Property 1.
    \item[(II)] Propert 2.
    \item[(III)] Final thing.
\end{itemize}

\newpage

\begin{enumerate}
    \item Property 1.
    \item Propert 2.
    \item Final thing.
\end{enumerate}


% This is how to create a comment in LaTeX
% The $ here puts us into "math mode" and LaTeX will convert this properly.
$\alpha$


a=b=c

\begin{itemize} \setlength\itemsep{0em}
    \item size test
    \item {\tiny size test}
    \item {\large size test}
    \item {\Large size test}
     \item {\LARGE size test}
     \item {\huge size test}
     \item {\bf size test}
     \item {\it size test}
     \item \underline{testing still}
     \item $\mathbf{x}$
\end{itemize}



% Here \[ will also put is into "display style math mode" where the math is centered and more pronounced.
\[ a=b=c=\sum_{i=1}^n \alpha_i \beta^i \]

$\displaystyle a=b=c=\sum_{i=1}^n \alpha_i \beta^i $\\

$\emptyset$\\
$\epsilon$

$ a=b=c=\sum_{i=1}^n \alpha_i \beta^i $

% Matrix with square brackets
\[
A = \begin{bmatrix}
1 & 2 \\
3 & 4
\end{bmatrix} \in \mathbb{R}^{2\times 2} 
\]




%Display math mode equation
\[ 
(1-x)^3 = (1-x)(1-2x+x^2) = 1-2x+x^2 -x +2x^2 -x^3
\]


%desplay math mode align.  In the align environment the & symbols will line up, so here the = all line up.  The \\ is a line break.  The * here is just to remove any number labels to the equations.
\begin{align*}
(1-x)^3 &= (1-x)(1-2x+x^2)\\
        &= 1-2x+x^2 -x +2x^2 -x^3    
\end{align*}

$a=b and b=c$ \\

$a=b\text{ and }b=c$\\


$a=b$ and $b=c$


%here is the theorem environment
\begin{theorem}
Everything above QED is true
\end{theorem}
%here is the proof environment
\begin{proof}
QED
\end{proof}


\[ |z-a_{ii}| \leq  \sum_{i} a_i\] 

%sections and subsections can be created like this.  The * here will remove a number associated to the section.
\section*{Background}
\subsection*{Graphs}
\subsection*{Matrix Analysis}

\section*{Adjacency Matrices}


\end{document}
